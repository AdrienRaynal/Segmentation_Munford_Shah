% Ces trois commandes permettent de limiter les coupures de mots en fin de ligne
\hyphenpenalty=10000
\tolerance=2000
\emergencystretch=10pt


% Changement de la police
\usepackage{lmodern}


% Pour avoir les références dans la table des matières
\usepackage[nottoc]{tocbibind} % A utilisé que dans les classes book, report, article, proc et ltxdoc


% Définitions des environnements mathématiques
% On peut changer ici le style de présentation des définitions et compagnie sans avoir
% à le faire partout dans le document
{
\theorembodyfont{\normalfont}
\theorempostwork{\smallskip}
\newtheorem{remarque}{Remarque}[section]
\theoremseparator{ -}
\newtheorem{definition}{Définition}[section]
\newtheorem{theoreme}{Théorème}[section]
\newtheorem{proposition}{Proposition}[section]
\newtheorem{corollaire}{Corollaire}[section]
\newtheorem{lemme}{Lemme}[section]
}
{
\theorembodyfont{\normalfont}
\theoremstyle{break}
\theoremsymbol{$\Box$}
\theorempostwork{\smallskip}
\theoremseparator{ :}
\newtheorem*{preuve}{Preuve}
\newtheorem*{preuvealgo}{Preuve de l'algorithme}
}
%\numberwithin{equation}{section}


% Définition d'un nouvel environnement pour afficher du code de manière jolie.
\newcounter{nalg}[section]
\newcommand\keywordstyle[1]{{\color{black}\bfseries{#1}}}
\renewcommand{\thenalg}{\thesection .\arabic{nalg}}
\DeclareCaptionLabelFormat{algocaption}{\textbf{Algorithme \thenalg}}
\lstnewenvironment{algorithm}[1][]
{
	\refstepcounter{nalg} % On augmente de un le compteur
    \captionsetup{labelformat=algocaption, labelsep=colon}
    \lstset{ % Définie le style de l'environnement
        mathescape=true,
        breaklines=true,
        escapechar=|,
        columns=fullflexible,
        keepspaces=true,
        frame=tl,
        tabsize=4,
        keywordstyle=\color{black}\bfseries,
        keywords={, Sortie, Renvoyer, Si, Alors, Sinon, Pour, Tout, Tant, Que, Fin, Fois, Faire, }
        xleftmargin=.04\textwidth,
        literate=
			{à}{{\`a}}1
			{ç}{{\c{c}}}1
			{é}{{\'e}}1
			{è}{{\`e}}1
			{ê}{{\^e}}1
			{î}{{\^i}}1
			{ô}{{\^o}}1
			{Entrée}{{\keywordstyle{Entr\'ee}}}6
			{Début}{{\keywordstyle{D\'ebut}}}5
			{Répéter}{{\keywordstyle{R\'ep\'eter}}}7
			,
        #1 % Permet d'ajouter d'autre paramètre à l'environnement, comme le titre et un label de référence
    }
}
{\bigskip}


% Définition d'une nouvelle commandes pour les images tikz
% Argument 1 (optionnel) : postion de l'image (center, right, ...), par défaut center
% Argument 2 : nom du fichier de l'image
% Argument 3 : légende de l'image
% Argument 4 : nom du label pour citer l'image avec \cite, laisser vide si pas voulu
\newcommand{\tikzimage}[4][center]{
\begin{figure}[H]
	\begin{#1}
	\captionsetup{type=figure, belowskip=-20pt}
	\input{#2}
	\begin{center}
	\caption{#3}
	\ifthenelse{\equal{#4}{}}{}{\label{#4}}
	\end{center}
	\end{#1}
\end{figure}
}


%%%%%%%%%%%%%%%%%%%%%%%%%%%%%%%%%%%%%%%%%%%%%%%%%%%%%%%%%%%%%%%%%%%
% Commandes pour les maths


% Pour afficher R^n, l'argument n peut être laissé vide
\newcommand{\R}[1][]{\ensuremath{\ifthenelse{\equal{#1}{}}
{\mathbb{R}}
{\mathbb{R}^{#1}}
}\xspace}
% Pour afficher Z^n, l'argument n peut être laissé vide
\newcommand{\Z}[1][]{\ensuremath{\ifthenelse{\equal{#1}{}}
{\mathbb{Z}}
{\mathbb{Z}^{#1}}
}\xspace}
% Pour afficher N^n, l'argument n peut être laissé vide
\newcommand{\N}[1][]{\ensuremath{\ifthenelse{\equal{#1}{}}
{\mathbb{N}}
{\mathbb{N}^{#1}}
}\xspace}


% Pour afficher la fonction de hauteur
% Argument 1 : nom du pavage associé
% Argument 2 : point auquel on évalue (peut être laissé vide)
\newcommand{\fcth}[2]{\ensuremath{\ifthenelse{\equal{#2}{}}
{h_{#1}}
{h_{#1}(#2)}
}}

% Fonctions minimales et maximales
\newcommand{\fcthmin}[2]{\fcth{{#1}_{min}}{#2}}
\newcommand{\fcthmax}[2]{\fcth{{#1}_{max}}{#2}}


% Fonction pour le nom donné au quadrillage de R^2
\newcommand{\quadrillage}{\ensuremath{\Lambda}\xspace}


% Fonction pour le nom donné à l'ensemble des pavages
% Argument 1 : nom du polygone dont on cite l'ensemble des pavages
\newcommand{\pavages}[1]{\ensuremath{\mathcal{T}_{#1}}}

% Commande pour le P de probabilité
\renewcommand{\P}[1][]{\ensuremath{\ifthenelse{\equal{#1}{}}
{\mathbb{P}}
{\mathbb{P}(#1)}
}\xspace}
